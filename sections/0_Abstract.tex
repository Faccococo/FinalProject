% !Mode:: "TeX:UTF-8"
% !TEX program  = xelatex
% \begin{中文摘要}{\LaTeX ;接口}
%   笔者见到的毕业论文模板,大多是以文类的形式,少部分以宏包的形式,并且在模板中大多掺杂着各式各样的例子(除了维护频率高的模板),导致模板文件使用了大部分与形式格式不相关的内容,代码量巨大文档欠缺且不容易修改,出现问题需要查看宏包或者文类的源代码。于是,秉着仅提供实现最基本要求的理念,重构了之前所写的 \TeX\ 形式。由于第二年使用该模板,所以设计出的模板接口不能保证以后不发生重大变动,一切以文档为主。毕竟学校在发展初期,各类文件都在日渐完善,前几年时,学校标志及名称还发生变化,同时毕业论文的样式也发生了重大变化。但是可以保证的是,模板提供的接口均为中文形式\footnote{使用 \hologo{XeLaTeX} 特性,一方面增加辨识度,另一方面不拘泥于英文命名的规则。当然此举也有些许弊端,在此就不过多展开。},并且至少更新到 2021 年,也就是笔者毕业。模板这种东西不能保证一劳永逸,一方面学校的标准制度都在发生着改变,另一方面 \hologo{LaTeX} 的宏包也在发生着改变,早先流行的宏包可能几年后就被“淘汰”掉。因此,您的使用与反馈是我不断更新的动力,希望各位不吝赐教。
% \end{中文摘要}

% \begin{英文摘要}{LaTeX, Interface}
%   \lipsum[1]
% \end{英文摘要}

\begin{英文摘要}{VR, Monocular Distance Measurement, Off-axis Projection, Kalman Filter, OpenCV, Unity}

In the realm of virtual reality (VR), achieving a truly immersive experience typically necessitates the use of cumbersome headgear, which can impede user comfort and accessibility. This project introduces a novel VR system designed to circumvent these limitations by employing monocular ranging and off-axis projection techniques. The system transforms a standard display into a dynamic "window" into a virtual environment, adjusting the visual output based on the viewer’s position. This adjustment simulates real-world perspective shifts, enhancing the realism of the virtual experience without the need for specialized equipment. We detail the development of this technology, emphasizing its lightweight implementation and potential for broad application, from entertainment to educational training. Preliminary evaluations demonstrate the system’s efficacy in providing a compelling and interactive user experience, suggesting its viability as a practical alternative to traditional VR solutions. The main contributions of this research including a image correction algorithm for curved screen and a monocular head-tracking algorithm.

% This technology's ingenuity lies in its ability to surmount both the physical and fiscal impediments associated with standard VR modalities, while also facilitating a more organic and intuitive user interaction by meticulously tracking the user’s facial and eye movements. The prospective applications of this methodology are extensive, spanning entertainment and gaming to sectors like education, healthcare, and industrial design, offering an immersive digital milieu accessible without the dependency on supplementary apparatus.
\end{英文摘要}
