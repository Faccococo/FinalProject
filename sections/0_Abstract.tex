% !Mode:: "TeX:UTF-8"
% !TEX program  = xelatex
% \begin{中文摘要}{\LaTeX ;接口}
%   笔者见到的毕业论文模板,大多是以文类的形式,少部分以宏包的形式,并且在模板中大多掺杂着各式各样的例子(除了维护频率高的模板),导致模板文件使用了大部分与形式格式不相关的内容,代码量巨大文档欠缺且不容易修改,出现问题需要查看宏包或者文类的源代码。于是,秉着仅提供实现最基本要求的理念,重构了之前所写的 \TeX\ 形式。由于第二年使用该模板,所以设计出的模板接口不能保证以后不发生重大变动,一切以文档为主。毕竟学校在发展初期,各类文件都在日渐完善,前几年时,学校标志及名称还发生变化,同时毕业论文的样式也发生了重大变化。但是可以保证的是,模板提供的接口均为中文形式\footnote{使用 \hologo{XeLaTeX} 特性,一方面增加辨识度,另一方面不拘泥于英文命名的规则。当然此举也有些许弊端,在此就不过多展开。},并且至少更新到 2021 年,也就是笔者毕业。模板这种东西不能保证一劳永逸,一方面学校的标准制度都在发生着改变,另一方面 \hologo{LaTeX} 的宏包也在发生着改变,早先流行的宏包可能几年后就被“淘汰”掉。因此,您的使用与反馈是我不断更新的动力,希望各位不吝赐教。
% \end{中文摘要}

% \begin{英文摘要}{LaTeX, Interface}
%   \lipsum[1]
% \end{英文摘要}

\begin{英文摘要}{VR, Monocular Distance Measurement, Off-axis Projection, Kalman Filter, OpenCV, Unity}
Amid the brisk pace of technological evolution, Virtual Reality (VR) has emerged as a focal point due to its capacity to deliver immersive and interactive experiences. Traditional VR frameworks, however, have typically necessitated the use of specialized head-mounted displays (HMDs). This requirement not only circumscribes their practicality but also impacts user comfort and the viability of extended engagement. This report delineates an innovative approach to implementing glasses-free VR, aimed at severing the reliance on conventional HMDs through the adoption of monocular distance measurement and off-axis projection techniques, thus amplifying user accessibility and involvement.

The principal technological underpinnings of this project involve the extraction of facial keypoints (executed via OpenCV), the determination of facial positions using PnP algorithms, and the employment of a Kalman filter to diminish the effects of keypoint jitter. Through off-axis projection, we successfully render scenes crafted in Unity in a manner that engenders a VR experience devoid of any requisite headgear.

% This technology's ingenuity lies in its ability to surmount both the physical and fiscal impediments associated with standard VR modalities, while also facilitating a more organic and intuitive user interaction by meticulously tracking the user’s facial and eye movements. The prospective applications of this methodology are extensive, spanning entertainment and gaming to sectors like education, healthcare, and industrial design, offering an immersive digital milieu accessible without the dependency on supplementary apparatus.
\end{英文摘要}
