\section {Introduction}
Amid the brisk pace of technological evolution, Virtual Reality (VR) has emerged as a focal point due to its capacity to deliver immersive and interactive experiences. Traditional VR frameworks, however, have typically necessitated the use of specialized head-mounted displays (HMDs). This requirement not only circumscribes their practicality but also impacts user comfort and the viability of extended engagement. This report delineates an innovative approach to implementing glasses-free VR, aimed at severing the reliance on conventional HMDs through the adoption of monocular distance measurement and off-axis projection techniques, thus amplifying user accessibility and involvement.

The principal technological underpinnings of this project involve the extraction of facial keypoints (executed via OpenCV), the determination of facial positions using PnP algorithms, and the employment of a Kalman filter to diminish the effects of keypoint jitter. Through off-axis projection, we successfully render scenes crafted in Unity in a manner that engenders a VR experience devoid of any requisite headgear.

This technology's ingenuity lies in its ability to surmount both the physical and fiscal impediments associated with standard VR modalities, while also facilitating a more organic and intuitive user interaction by meticulously tracking the user’s facial and eye movements. The prospective applications of this methodology are extensive, spanning entertainment and gaming to sectors like education, healthcare, and industrial design, offering an immersive digital milieu accessible without the dependency on supplementary apparatus.
