% !Mode:: "TeX:UTF-8"
% !TEX program  = xelatex

\begin{英文摘要}{VR, Head-Track, Off-axis Projection, Kalman Filter, OpenCV, Unity}

% Immersive virtual reality (VR) systems have demonstrated considerable potential across a spectrum of applications ranging from entertainment to education. Existing VR solutions typically necessitate the use of specialized headgear, thereby limiting their accessibility and convenience.

Immersive 3D systems have demonstrated considerable potential across a spectrum of applications ranging from entertainment to education. Existing 3D solutions typically necessitate the use of specialized glasses and professional photographic devices, thereby limiting their accessibility and convenience. In this work, we introduce an innovative 3D system that employs monocular ranging and off-axis projection techniques to facilitate an immersive experience without the encumbrance of cumbersome equipment. Our system dynamically adjusts the virtual scene based on the viewer's position, thereby providing a "window" into the virtual world that mimics the real-world illusion.

To achieve this objective, we propose a lightweight and cost-effective head tracking solution that leverages YuNet facial keypoint detection, integrated with the Enhanced Perspective-n-Point (ePnP) algorithm and Kalman filtering, to realize effective monocular head tracking. Furthermore, the study introduces a novel algorithm for optimizing image display on curved screens. This algorithm involves the horizontal segmentation of images and adapts each segment to the curvature, optimizing the image presentation on curved screens. This method ensures that the displayed images maintain their intended appearance, thereby enhancing the realism of the 3D experience. Our approach has been fine-tuned with a combination of real-time computational models and mobile-friendly implementations to ensure efficiency and responsiveness.

% This technology's ingenuity lies in its ability to surmount both the physical and fiscal impediments associated with standard VR modalities, while also facilitating a more organic and intuitive user interaction by meticulously tracking the user’s facial and eye movements. The prospective applications of this methodology are extensive, spanning entertainment and gaming to sectors like education, healthcare, and industrial design, offering an immersive digital milieu accessible without the dependency on supplementary apparatus.
\end{英文摘要}

\begin{中文摘要}{VR, Head-Track, Off-axis Projection, Kalman Filter, OpenCV, Unity}

% 在虚拟现实(VR)领域,实现真正沉浸式体验通常需要使用笨重的头戴设备,这会影响用户的舒适度和可及性。本项目引入了一种新颖的VR系统,旨在通过单目测距和偏轴投影技术规避这些限制。该系统将标准显示屏转变为通向虚拟环境的动态“窗口”,根据观众的位置调整视觉输出。这种调整模拟了现实世界中的视角变化,在无需专用设备的情况下增强了虚拟体验的真实性。我们详细介绍了该技术的开发,强调其轻量化实现和广泛应用的潜力,从娱乐到教育培训等多个领域。初步评估显示,该系统在提供引人入胜和互动的用户体验方面效果显著,表明其作为传统VR解决方案的实际替代方案的可行性。


沉浸式3D系统在娱乐和教育等广泛应用领域展现出巨大的潜力。现有的3D解决方案通常需要使用专业的摄影设备和特制眼镜,从而限制了其普及性和便利性。在本研究中,我们引入了一种创新的3D系统,该系统采用单目测距和偏轴投影技术,无需笨重设备即可提供沉浸式体验。我们的系统能够根据观众的位置动态调整虚拟场景,从而提供一个“窗口”式的虚拟世界,模拟现实世界的错觉。

为了实现这一目标,我们提出了一种轻量且成本效益高的头部追踪解决方案,该方案利用YuNet面部关键点检测,与增强型透视n点(ePnP)算法和卡尔曼滤波集成,实现有效的单目头部追踪。此外,本研究还引入了一种优化曲面屏幕图像显示的新算法。该算法通过对图像进行水平分割,并根据曲率调整每个分段,从而优化图像在曲面屏幕上的显示效果。此方法确保显示的图像保持其预期外观,从而增强3D体验的真实感。我们的方法结合了实时计算模型和移动设备友好的实现,确保了效率和响应能力。


\end{中文摘要}
