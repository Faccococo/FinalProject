% !Mode:: "TeX:UTF-8"
% !TEX program  = xelatex
% \section{免责声明}
% \begin{enumerate}[label={\alph*)}]
%     \item 本模板的发布遵守 \LaTeX\ Project Public License,使用前请认真阅读协议内容。
%     \item 南方科技大学教学工作部只提供毕业论文写作指南,不提供官方模板,也不会授权第三方模板为官方模板,所以此模板仅为写作指南的参考实现,不保证格式审查老师不提意见. 任何由于使用本模板而引起的论文格式审查问题均与本模板作者无关。
%     \item 任何个人或组织以本模板为基础进行修改,扩展而生成的新的专用模板,请严格遵守 \LaTeX\ Project Public License 协议。由于违犯协议而引起的任何纠纷争端均与本模板作者无关。
% \end{enumerate}
\section {Introduction}
Amid the brisk pace of technological evolution, Virtual Reality (VR) has emerged as a focal point due to its capacity to deliver immersive and interactive experiences. Traditional VR frameworks, however, have typically necessitated the use of specialized head-mounted displays (HMDs). This requirement not only circumscribes their practicality but also impacts user comfort and the viability of extended engagement. This report delineates an innovative approach to implementing glasses-free VR, aimed at severing the reliance on conventional HMDs through the adoption of monocular distance measurement and off-axis projection techniques, thus amplifying user accessibility and involvement.

The principal technological underpinnings of this project involve the extraction of facial  landmarks (executed via OpenCV), the determination of facial positions using PnP algorithms, and the employment of a Kalman filter to diminish the effects of keypoint jitter. Through off-axis projection, we successfully render scenes crafted in Unity in a manner that engenders a VR experience devoid of any requisite headgear.

This technology's ingenuity lies in its ability to surmount both the physical and fiscal impediments associated with standard VR modalities, while also facilitating a more organic and intuitive user interaction by meticulously tracking the user’s facial and eye movements. The prospective applications of this methodology are extensive, spanning entertainment and gaming to sectors like education, healthcare, and industrial design, offering an immersive digital milieu accessible without the dependency on supplementary apparatus.
